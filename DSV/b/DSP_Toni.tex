\documentclass[10pt,a4paper]{article}
\usepackage[utf8]{inputenc}
\usepackage[ngerman]{babel}
\usepackage[T1]{fontenc}
\usepackage{amsmath}
\usepackage{amsfonts}
\usepackage{amssymb}
\usepackage{graphicx}
\usepackage{lmodern}
\usepackage{physics}
\usepackage[left=1cm,right=1cm,top=1.5cm,bottom=1.2cm]{geometry}
\usepackage{siunitx}
\usepackage{fancyhdr}
\usepackage{enumerate}
\usepackage{mhchem}
\usepackage{mathtools}
\usepackage{graphicx}
\usepackage{float}
\usepackage[table]{xcolor}
\usepackage{mdframed}
\usepackage{csquotes}
\usepackage{trfsigns}
\usepackage{capt-of}
\usepackage{minted}

\sisetup{locale=DE}
\sisetup{per-mode = symbol-or-fraction}
\sisetup{separate-uncertainty=true}
\DeclareSIUnit\year{a}
\DeclareSIUnit\clight{c}
\mdfdefinestyle{exercise}{
	backgroundcolor=black!10,roundcorner=8pt,hidealllines=true,nobreak
}

\begin{document}
\twocolumn
\pagestyle{fancy}
\lhead{DSP Formelsammlung, \today}
\rhead{Sedlmeier Toni \& Kirschner Christoph}
\section{Zahlenformate}
\begin{center}
  \includegraphics[width=.5\textwidth]{./img/dynamic.png}
\end{center}
%%%%%%%%%%%%%%%%%%%%%%%%%%%%%%%%%%%%%%%%%%%%% Zahlenformate %%%%%%%%%%%%%%%%%%%%%%%%%%%%%%%%%%%%%%%%%%%%%%%%%%%%%
  \subsection{Zweierkomplement}
  Umwandlung: Bsp. 8-Bit $(-4)_{10}$ (Funktioniert in beide Richtungen)
  \begin{itemize}
      \item Vorzeichen Ignorieren $(4)_{10} = (0000 0100)_2$
      \item Bits Invertieren $(0000 \ 0100)_2 \ \rightarrow (1111\ 1011)_2$
      \item Eins Addieren $(1111\ 1011)_2 + (0000\ 0001)_2 = (1111\ 1100)_2$
  \end{itemize} 
  \subsection{Fixed Point (unsigned)}
Qk.l mit k = Vorkomma und l = Nachkomma
  \begin{mdframed}[style=exercise]
    \begin{align}
        x_{(10)} = \sum_{i=0}^{k-1} b_i\cdot 2^i + \sum_{j=-l}^{-1} b_j\cdot 2^j
    \end{align}
  \end{mdframed}
Bsp Q4.5 $a=(0101 0110)_2$ \\
$\textcolor{red}{0}\cdot 2^3+\textcolor{red}{1}\cdot 2^2+\textcolor{red}{0}\cdot 2^{1}+\textcolor{red}{1}\cdot 2^0+\textcolor{red}{0}\cdot 2^{-1}+\textcolor{red}{1}\cdot 2^{-2}+\textcolor{red}{1}\cdot 2^{-3}+\textcolor{red}{0}\cdot 2^{-4}=5.375$

  \subsection{Fixed Point (signed)}
Bsp. Q3.3 $(100.001)_2$ 
  \begin{itemize}
      \item Vorzeichen Merken (\textcolor{red}{1}00.001) $\rightarrow \ \textcolor{red}{-1}$
      \item Bits Invertieren $(100 \ 001)_2 \ \rightarrow (011\ 110)_2$
      \item $1\cdot 2^{-k}$ Addieren $(011\ 110)_2 + (000\ 001)_2 = (011\ 111)_2 = -3.875$
  \end{itemize} 

%%%%%%%%%%%%%%%%%%%%%%%%%%%%%%%%%%%%%%%%%%%%% Filter in c %%%%%%%%%%%%%%%%%%%%%%%%%%%%%%%%%%%%%%%%%%%%%%%%%%%%%
%%%%%%%%%%%%%%%%%%%%%%%%%%%%%%%%%%%%%%%%%%%%%  FIR %%%%%%%%%%%%%%%%%%%%%%%%%%%%%%%%%%%%%%%%%%%%%%%%%%%%%
\section{Filter in C}
\subsection{FIR}
\subsubsection{Blockschaltbild und math. Zusammenhänge:}
\begin{center}
  \includegraphics[width=.3\textwidth]{./img/FIR_Blockschaltbild.png}  
\end{center}

  \begin{mdframed}[style=exercise]
    \begin{align}
        A(0)y(k)=\sum_{i=0}^{N} B(i) x(k-i) 
    \end{align}
  \end{mdframed}

\subsubsection{Realisierung des FIR in C (Brute-Force):} 
\begin{minted}[fontsize=\scriptsize]{c}
void fir()
{
    CodecDataIn.UINT = ReadCodecData();     // get input data samples
    int i;
    xLeft[0] = CodecDataIn.Channel[LEFT];   // current input value
    yLeft = 0 ;                             // initialize the output
    for ( i = 0 ; i <= N ; i++) {           // x is length N+1
        yLeft += xLeft[i] * B[i];           // perform the dot-product
    }
    for ( i = N ; i > 0 ; i--) {
        xLeft[i] = xLeft[i-1];              // shift for the next input
    }
    CodecDataOut.Channel[LEFT] = yLeft;     // output the value
}
\end{minted}

\begin{itemize}
    \item nach jedem dot-product ineffizientes Shiften von xLeft
    \item Problem kann mit Ringbuffer behoben werden
\end{itemize}

\subsubsection{FIR Ringbuffer}
\textbf{Vorteil: } Kein Kopieren nötig
\begin{center}    
  \includegraphics[width=.5\textwidth]{./img/ring.png}  
\end{center}  


\textbf{Realisierung des FIR-Ringpuffers in C: }
\begin{minted}[fontsize=\scriptsize]{c}
 void firRing()
 {
     float xLeft[N+1];
     float pLeft = xLeft;
     *pLeft = CodecDataIn.Channel[LEFT]; // store input value
     output = 0 ;                        // set up for channel
     p = pLeft ;                         // save current sample pointer
     if(++pLeft > &xLeft[N])             // update pointer
     pLeft = xLeft;                      // and store
     for(i = 0 ; i <= N ; i++) {         // do LEFT channel FIR
         output+=(*p--) * B[i];          // multiply and accumulate
         if(p < &xLeft[0])               // check for pointer wrap around
         p = &xLeft[N];
     }
     CodecDataOut.Channel[LEFT]=output; // store filtered value
 }
 \end{minted}

\newpage 
%%%%%%%%%%%%%%%%%%%%%%%%%%%%%%%%%%%%%%%%%%%%%  IIR %%%%%%%%%%%%%%%%%%%%%%%%%%%%%%%%%%%%%%%%%%%%%%%%%%%%%
\subsection{IIR}
\subsubsection{Blockschaltbild und math. Zusammenhänge:}

\textbf{Direktform I:}
\begin{center}    
  \includegraphics[width=.5\textwidth]{./img/Direktform_1_IIR.png}  
\end{center}  

\textbf{Direktform II:}
\begin{center}
  \includegraphics[width=.5\textwidth]{./img/Direktform_2_IIR.png}
\end{center}

\textbf{3. Kanonische Form:}
\begin{center}
  \includegraphics[width=.5\textwidth]{./img/Kanonische_Form_3_IIR.png}
\end{center}

\textbf{4. Kanonische Form:}
\begin{center}
  \includegraphics[width=.5\textwidth]{./img/Kanonische_Form_4_IIR.png}
\end{center}

  \begin{mdframed}[style=exercise]
    \begin{align}
        A(0)y(k)=\sum_{i=0}^{N} B(i) x(k-i) - \sum_{i=0}^{N} A(i) y(k-i)
    \end{align}
  \end{mdframed}

\subsubsection{Realisierung des IIR in C:}

\begin{minted}[fontsize=\scriptsize]{c}
void iir()
{
    xLeft[0]  = CodecDataIn.Channel[LEFT] + CodecDataIn.Channel[RIGHT];
    yLeft[0] = 0;                           // initialize the output value

    for (i = 0; i <= N; i++){
        yLeft[0] += B[i] * xLeft[i];
    }
    for (i = 1; i <= N; i++){
        yLeft[0] -=  A[i] * yLeft[i];
    }
    for (i = N; i > 0; i--)
    {
        xLeft[i] = xLeft[i-1];
        yLeft[i] = yLeft[i-1];
    }
    CodecDataOut.Channel[LEFT]  = yLeft[0]; // output the filtered value
}
\end{minted}
% \scalebox{0.85}{
%     \begin{center}
%     \begin{tabular}{ | c | c | c | }
% \cline{1-3}
%         & Zeitbereich & Spektralbereich \\
% \cline{1-3}
%         Linearität & $a\cdot x_1(k) b\cdot x_2(k)$ & $a\cdot X_1(n) +b\cdot X_2(n)$ \\
%     \end{tabular}
%     \end{center}

%%%%%%%%%%%%%%%%%%%%%%%%%%%%%%%%%%%%%%%%%%%%%  Notch %%%%%%%%%%%%%%%%%%%%%%%%%%%%%%%%%%%%%%%%%%%%%%%%%%%%%
\subsection{Notch-IIR-Filter}
\subsubsection{Allgemeines:}
\textbf{SOS-IIR-Notch-Filter (Beispiel):}
\begin{center}
  \includegraphics[width=.5\textwidth]{./img/Notch_Filter_SOS.png}
\end{center}

  \begin{mdframed}[style=exercise]
    \begin{align}
  H(z) = k \frac{(z-\beta_1)(z-\beta_2)}{(z-\alpha_1)(z-\alpha_2)} 
       = k \frac{1-2\cos(\omega_0)z^{-1}+z^{-2}}{1-2r\cos(\omega_0)z^{-1}+r^2 z^{-2}}
    \end{align}
  \end{mdframed}
$f_0$ = Kerbfrequenz, $f_s$ = Abtastfrequenz \\
$B_{3dB}$ = Kerbbreite 
  \begin{mdframed}[style=exercise]
    \begin{align}
      k = \frac{1-2r\cos(\Omega_0)+r^2}{1-2\cos(\Omega_0)+1}\\
      \Omega_0 = 2\pi \frac{f_0}{f_s}\\
      r=1-(\frac{B_{3dB}}{f_s})\pi
    \end{align}
  \end{mdframed}

\subsubsection{Anwendungsbeispiel:}
\begin{enumerate}
  \item Ermitteln des spektralen Maximum in durch die FFT generierten Daten
  \item Ermitteln der entsprechenden Störfrequenzen
  \item Errechnen der Koeffizienten des Notch-Filters mit der Störfrequenz als Kerb-Frequenz
  \item Filtern des diskretisierten Signales mit errechnetem Filter
  \item Ausgabe der bearbeiteten Audio-Sequenz
\end{enumerate}

\section{Spektralschätzung}
\subsection{Fensterfunktionen}

\begin{center}
  \includegraphics[width=.5\textwidth]{./img/Fensterfunktionen_Vergleich.png}
\end{center}

\section{Oszillatoren / Signalgeneratoren}
%%%%%%%%%%%%%%%%%%%%%%%%%%%%%%%%%%%%%%%%%%%%%  IIR-Oszillator %%%%%%%%%%%%%%%%%%%%%%%%%%%%%%%%%%%%%%%%%%%%%%%%%%%%%
\subsection{IIR-Oszillator (Digital Resonator)}
\subsubsection{Allgemeines:}
\begin{itemize}
  \item Oszillator auf Basis einer z-Transformation
  \item Anregung des Systems mit Impuls bei $k=0$
  \item System mit Polen auf Einheitskreis\\
  $\rightarrow$ Grenzstabiles System\\
  $\rightarrow$ System schwingt mit konst. Frequenz
\end{itemize}

\textbf{Vorgehen:}
\begin{enumerate}
  \item z-Transformation des kontinuierlichen Systems
  \item Ausmultiplizieren von $H(z)$ mit $Y(z)$ bzw. $X(z)$
  \item Anwenden von \\
  $Y(z)\cdot z^{-1}=y[n-1]$ bzw. $x(z)\cdot z^{-1}=x[k-1]$
  \item Auflösen nach y(k)
  \item Impuls als Eingangssignal $x[k] = [1, 0, 0, ...]$
\end{enumerate}

\textbf{Vorteil:}
\begin{itemize}
  \item Minimaler Verbrauch von Speicher und Rechenleistung
  \item Anpassbarkeit an beliebige Funktionen
  \item Hohe Auflösung und Flexibilität
\end{itemize}  

\textbf{Nachteil:}
\begin{itemize}
  \item Frequenz muss vor Start festgelegt werden
  \item durch Quantisierung kann Pol aus dem Einheitskreis rutschen und instabil werden
  \item Oszillator muss einschwingen
\end{itemize}  

  \begin{mdframed}[style=exercise]
    \begin{align}
    sin(\Omega_0 k) \leftrightarrow \frac{sin(\Omega_0)z^{-1}}{1-2\cos(\Omega_0)z^{-1}+z^{-2}}\\
    y(n) = sin(\Omega_0)x(n-1)+2cos(\Omega_0)y(n-1)-y(n-2)
    \end{align}
  \end{mdframed}

\subsubsection{Realisierung des IIR-Oszillators in C:}
  \begin{minted}[fontsize=\scriptsize]{c}
    enum lrtype {LEFT, RIGHT};
    volatile union {unsigned UINT; short Channel[2];}
    CodecDataIn, CodecDataOut;
    float fDesired = 1000; // your desired signal frequency
    float A = 32000; // your desired signal amplitude
    float pi = 3.1415927; // value of pi
    float theta;// digital frequency (omega0 in textbook)
    float y[3] = {0, 1, 0}; // the last 3 output values.
    unsigned fs = 48000; // sample frequency

    void isr_resonator(){
      CodecDataIn.UINT = ReadCodecData(); // get input data samples
      theta = 2 * pi * fDesired / fs; // calc. the digital frequency
      y[0] = 2 * cosf(theta) * y[1] - y[2]; // calculate the output
      y[2] = y[1]; // prepare for the next ISR
      y[1] = y[0]; // prepare for the next ISR
      CodecDataOut.Channel[ LEFT] = A * sinf(theta) * y[0]; // just scale
      CodecDataOut.Channel[RIGHT] = CodecDataOut.Channel[LEFT];
    }

    WriteCodecData(CodecDataOut.UINT);  
  \end{minted}

\subsection{DDS-Oszillator}
\subsubsection{Allgemeines:}
\begin{itemize}
  \item Direct Digital Synthesizer
  \item Errechnen des Funktionsverlaufs
  \item Verwendung von Accumulator und Modulo-Operator
  \item sin() kann berechnet oder mittels Lookup-Table realisiert werden
  \item Frequenzauflösung abhängig von Wortbreite des Phasenakkumulators $\frac{f_{clk}}{2^{N_{Bits}}}$
\end{itemize}

\textbf{Vorteil:}
\begin{itemize}
  \item Kann einfach bei FPGA verwendet werden
  \item Robuster Phasen- oder Frequenzwechsel mit sofortiger Wirkung
  \item kontinuierliche Signalform
  \item keine Einschwingzeit
\end{itemize}  

\textbf{Nachteil:}
\begin{itemize}
  \item Höherer Speicher- und Rechenleistungs-Bedarf
  \item Frequenzauflösung abhängig im Wesentlichen von Auflösung der Lookup ab
\end{itemize}  

\begin{center}
  \includegraphics[width=.5\textwidth]{./img/DDS_sin().png}
\end{center} 

\begin{center}
  \includegraphics[width=.5\textwidth]{./img/sin_phaseincrement.png}
\end{center} 

  \begin{mdframed}[style=exercise]
    \begin{align}
        \varphi_{inc} = 2\pi\frac{f_0}{f_s}\\
        \varphi = \varphi + \varphi_{inc}\\
        \varphi_{inc} < \pi (sonst Aliasing)\\
        x(n)= Asin(n\varphi)
    \end{align}
  \end{mdframed}

\subsubsection{Realisierung des DDR-Oszillators in C:}
\begin{minted}[fontsize=\scriptsize]{c}
float A = 32000; //signal's amplitude 
float fDesired = 1000; // signal's frequency 
float phase = 0; // signal's initial phase 
float pi = 3.1415927; // value of pi 
float phaseIncrement; // incremental phase 
int fs = 48000; // sample frequency 

void sineGen_ISR(){
CodecDataIn.UINT = ReadCodecData();
// algorithm begins here 
phaseIncrement = 2*pi*fDesired/fs;
phase += phaseIncrement;
if (phase >= 2*pi) phase -= 2*pi;
// get input data samples
// calculate the phase increment 
// calculate the next phase 
// modulus 2*pi operation 
CodecDataOut.Channel[ LEFT] = A*sinf(phase); // scaled L output 
CodecDataOut.Channel[RIGHT] = A*cosf(phase); // scaled R output 
// algorithm ends here 
}

WriteCodecData(CodecDataOut.UINT);
\end{minted}


\subsubsection{Spezialfälle:}
\begin{enumerate}
  \item \textbf{Niquist-Frequenz: }$f=\frac{f_s}{2} \rightarrow \varphi_{inc} = \pi$\\
  $\rightarrow sin(n\cdot\pi) = 0$ bzw. $cos(n\cdot\pi) = [1, -1, 1, -1, ...]$
  \item $f=\frac{f_s}{4} \rightarrow \varphi_{inc} = \frac{\pi}{2}$\\
  $\rightarrow sin(n\cdot\pi) = [0, -1, 0, -1, ...]$\\ 
  $\rightarrow cos(n\cdot\pi) = [1, 0, -1, 0, ...]$
  \item \textbf{$f_s= N \cdot f$ (Ganzzahliges Vielfaches):} \\
  $\rightarrow$ nur N Werte müssen berechnet werden\\
  $\rightarrow$ $cos(\varphi_{inc}\cdot n)$ für n = 0, 1, ..., N
 
\end{enumerate}

%%%%%%%%%%%%%%%%%%%%%%%%%%%%%%%%%%%%%%%%%%%%%  Blocksignalverarbeitung %%%%%%%%%%%%%%%%%%%%%%%%%%%%%%%%%%%%%%%%%%%%%%%%%%%%%
\section{Blocksignalverarbeitung:}
\subsection{Dreifach-Puffer:}
\begin{center}
  \includegraphics[width=.5\textwidth]{./img/buffers.png}
\end{center}
\textbf{Allgemein:}
\begin{itemize}
  \item Kein Kopieren notwendig
  \item 3 unabhängige Buffer nötig
  \item 3 Pointer zeigen welcher Buffer befüllt, verarbeit bzw. entleert wird  
\end{itemize}

  \begin{center}
      \includegraphics[width=.5\textwidth]{./img/frame.png}
  \end{center}
ISR schreibt N samples nach buffer[fill\_index] und setzt ready\_index = fill\_index.\\ 
buffer[fill\_index] ist jetzt dran mit ProcessBuffer.
Jeder Frame generiert einen Interrupt.
  \begin{center}
      \includegraphics[width=.5\textwidth]{./img/buffers.png}
  \end{center}
  Block
\begin{itemize}
    \item fill\_index wird von ADC gefüllt
    \item dump\_index wird an DAC geschrieben
    \item ready\_index Buffer für Blocksignalverarbeitung
\end{itemize}

\subsubsection{Realisierung eines Dreifach-Puffers in C:}
\begin{minted}[fontsize=\scriptsize]{c}
#define BUFFER_LENGTH   	1024	// buffer length in samples 
#define NUM_BUFFERS         3
volatile float buffer[NUM_BUFFERS][2][BUFFER_LENGTH];

void ProcessBuffer()
// Processes the data in buffer[ready_index]
{
    volatile float *pL = buffer[ready_index][LEFT];
    volatile float *pR = buffer[ready_index][RIGHT];
    // Do the Process
    // ...
    buffer_ready = 0;   // means were done here
}

interrupt void Codec_ISR()
{                    
    static Uint8 fill_index = INITIAL_FILL_INDEX; // index to fill
    static Uint8 dump_index = INITIAL_DUMP_INDEX; // index to dump
    static Uint32 sample_count = 0; // current sample count in buffer

    // get input data samples
  	CodecDataIn.UINT = ReadCodecData();		
  	// IN
    buffer[fill_index][ LEFT][sample_count] = LEFT + RIGHT; // cropped
    buffer[fill_index][RIGHT][sample_count] = RIGHT + LEFT; // cropped
    // OUT
    CodecDataOut.channel[ LEFT] = buffer[dump_index][LEFT][sample_count];
    CodecDataOut.channel[RIGHT] = buffer[dump_index][RIGHT][sample_count];

    // update sample count and swap buffers when filled 
    if(++sample_count >= BUFFER_LENGTH) {
        sample_count = 0;
        ready_index = fill_index;
        if(++fill_index >= NUM_BUFFERS)
            fill_index = 0;
        if(++dump_index >= NUM_BUFFERS)
            dump_index = 0;
        if(buffer_ready == 1) // sover_run-Flag
            over_run = 1;
        buffer_ready = 1;
    }
	WriteCodecData(CodecDataOut.UINT);// send output data to  port
}
\end{minted}

\subsubsection{Blocksignalverarbeitung mit DMA}
\textbf{Vorteil: } Prozessor muss sich nicht mit Befüllen beschäftigen sondern kann verarbeiten\\
$\rightarrow$ Geschwindigkeitsvorteil

\begin{itemize}
    \item DMA kopiert Sample von ADC nach Eingangsbuffer
    \item DMA kopiert Processed von Ausgangsbuffer nach DAC
    \item DMA generiert Interrupt, wenn N Samples transfert wurden $\rightarrow$ Buffer-Swap
\end{itemize}

\begin{minted}[fontsize=\scriptsize]{c}
interrupt void EDMA_ISR()
{
    if(++ready_index >= NUM_BUFFERS)
        ready_index = 0;
    if(buffer_ready ==1) //buffer isnt processed in time
        over_run = 1;
    buffer_ready = 1; //buffer is now ready for processing
}
\end{minted}

\subsection{FIR mit Blocksignalverarbeitung}
\subsubsection{Allgemeines:}
Bsp. Ordnung Filter N = 4, Framesize = 8\\
  \begin{center}
      \includegraphics[width=.25\textwidth]{./img/firframe1.png}
      \includegraphics[width=.25\textwidth]{./img/firframe2.png}
  \end{center}
\textbf{Problem:} Die Start und Endzustände müssen jeweils berücksichtigt werden, um den FIR korrekt zu implementieren.\\ 
$\rightarrow$ Bei Frame-Übergängen müssen die N-1 letzten Werte des letzten Frames berücksichtigt werden.\\
$\rightarrow$ Audiotechnisch würde dies ein \grqq{}Knacken-\grqq{} und \grqq{}Klicken\grqq{} hervorrufen\\
\textbf{Lösung:} Der Puffer muss groß genug sein, um sowohl die Frame-Werte als auch die nötigen Randwerte speichern zu können (Framesize += N).
\\
\textbf{Latenz}: $\frac{2N}{f_clk}$ 2N: Eingangs-und Verarbeitungsbuffer

\begin{itemize}
    \item Left[N|FRAMESIZE], buffer[FRAMESIZE]
    \item Left[N:FRAMESIZE+N] = buffer[0:FRAMESIZE]
    \item buffer[0:FRAMESIZE] = Left * B (B wird \\drüber FRAMESIZE-mal drüber geschoben s.o)
    \item Left[0:N] = Left[FRAMESIZE:FRAMESIZE+N]
\end{itemize}

\subsubsection{Realisierung eines FIR mit Blocksignalverar. in C:}
\begin{minipage}{0.5\textwidth}
\begin{minted}[fontsize=\scriptsize]{c}
void ProcessBuffer()
{
    short *pBuf = buffer[ready_index];
    // extra buffer room for convolution "edge effects"
    // N is filter order from coeff.h
    static float Left[BUFFER_COUNT+N]={0}, Right[BUFFER_COUNT+N]={0};
    float *pL = Left, *pR = Right;
    float yLeft, yRight;
    int i, j, k;
    // offset pointers to start filling after N elements
    
    pR += N;
    pL += N;

    // extract data to float buffers
    for(i = 0; i < BUFFER_COUNT; i++) 
    { 
        *pR++ = *pBuf++;
        *pL++ = *pBuf++;
    }
    // reinitialize pointer before FOR loop
    pBuf = buffer[ready_index];

    // Implement FIR filter
    for(i=0; i < BUFFER_COUNT; i++) 
    {
        yLeft = 0; // initialize the LEFT output value
        yRight = 0; // initialize the RIGHT output value

        for(j=0,k=i+N; j <= N; j++,k--) 
        {
            yLeft += Left[k] * B[j]; // perform the LEFT dot-product
            yRight += Right[k] * B[j]; // perform the RIGHT dot-product
        }
        // pack into buffer after bounding (must be right then left)
        *pBuf++ = _spint(yRight * 65536) >> 16;
        *pBuf++ = _spint(yLeft * 65536) >> 16;
    }

    // save end values at end of buffer array for next pass
    // by placing at beginning of buffer array
    for(i=BUFFER_COUNT,j=0; i < BUFFER_COUNT+N; i++,j++) 
    {
        Left[j] = Left[i];
        Right[j] = Right[i];
    }
    buffer_ready = 0; // signal we are done
}
\end{minted}
\end{minipage}
\newpage
%%%%%%%%%%%%%%%%%%%%%%%%%%%%%%%%%%%%%%%%%%%%%  Ping-Pong %%%%%%%%%%%%%%%%%%%%%%%%%%%%%%%%%%%%%%%%%%%%%%%%%%%%%
\subsubsection{Allgemein:}
\textbf{Vorteile:}
\begin{itemize}
  \item Kein Kopieren zu Float-Arrays
  \item Robuste Buffer-Identifizierung, die Breakpoints unterstützt
\end{itemize}

\textbf{Nachteile:} Aufwendiges Schieben in dem Filterspeicher

  \begin{center}
      \includegraphics[width=.5\textwidth]{./img/pingpong.png}
  \end{center}

  \begin{itemize}
    \item gleiche Latenz (=Durchlaufzeit 2 Buffer)
    \item Ping-Pong einfacher zu verwalten mit DMA
  \end{itemize}

%%%%%%%%%%%%%%%%%%%%%%%%%%%%%%%%%%%%%%%%%%%%%  FFT %%%%%%%%%%%%%%%%%%%%%%%%%%%%%%%%%%%%%%%%%%%%%%%%%%%%%
\section{FFT}
\subsection{Allgemein: }
\begin{itemize}
  \item basiert auf dem \textbf{Devide-and-Conquer} Prinzip,
  \item Zwischenergebnisse werden wiederverwendet
  \item wesentliche Beschleunigung im Vergleich zur DFT
  \item \textbf{Ordnung DFT:} $N^2 \rightarrow$ \textbf{Ordnung FFT:} $N\cdot log_{2}(N)$
\end{itemize}
Mögliche Realisierungsformen:
\begin{itemize}
    \item Decimation in Frequency
    \item Decimation in Time 
\end{itemize}

\subsection{Decimation in Time (DIT)}
\textbf{Butterfly-Diagramm DIT-FFT radix-2 (N=8):}
\begin{center}
  \includegraphics[width=.3\textwidth]{./img/Decimation_in_Time.png}
\end{center} 

\textbf{1. Aufteilung in $Y(n) = Y_{even}(n)+Y_{odd}(n)$}
\begin{mdframed}[style=exercise]
  \begin{align}
      \textbf{Twiddle-Faktor: } w_N^{nk}&=e^{-j\frac{2\pi nk}{N}}\\
      w_N^{2nk}&=e^{-j\frac{4\pi nk}{N}}=w_{\frac{N}{2}}^{nk}
  \end{align}
\end{mdframed}

  \begin{mdframed}[style=exercise,font=\scriptsize]
    \begin{align}
        Y(n)&=\sum_{k=0}^{N/2-1} y(2k)w_N^{2nk}+\sum_{k=0}^{N/2-1} y(2k+1)w_N^{(2k+1)n}\\
        Y(n)&=\sum_{k=0}^{N/2-1} y(2k)w_N^{2nk}+w_N^{n}\sum_{k=0}^{N/2-1} y(2k+1)w_N^{2kn}\\
        Y(n)&=\sum_{k=0}^{N/2-1} y(2k)w_{\frac{N}{2}}^{nk}+w_N^{n}\sum_{k=0}^{N/2-1} y(2k+1)w_{\frac{N}{2}}^{nk}\\
    \end{align}
  \end{mdframed}
\textbf{2. Aufteilung in $Y(n) = Y_{left}(n)+Y_{right}(n)$}
  \begin{mdframed}[style=exercise,font=\scriptsize]
    \begin{align}
        Y(n)&=\sum_{k=0}^{N/2-1} y(2k)w_{\frac{N}{2}}^{nk}+w_N^{n}\sum_{k=0}^{N/2-1} y(2k+1)w_{\frac{N}{2}}^{nk}\\
        Y(n+\frac{N}{2})&=\sum_{k=0}^{N/2-1} y(2k)w_{\frac{N}{2}}^{(n+\frac{N}{2})k}+w_N^{n+\frac{N}{2}}\sum_{k=0}^{N/2-1} y(2k+1)w_{\frac{N}{2}}^{(n+\frac{N}{2})k}
    \end{align}
  \end{mdframed}
mit 
  \begin{mdframed}[style=exercise]
    \begin{align}
      \textbf{Twiddle-Faktor: } w_{\frac{N}{2}}^{(n+\frac{N}{2})k} &= w_{\frac{N}{2}}^{nk} \cdot \underbrace{w_{\frac{N}{2}}^{k\frac{N}{2}}}_{=1} = w_{\frac{N}{2}}^{nk}\\
        w_N^{n+\frac{N}{2}} &= w_N^n \cdot \underbrace{w^{\frac{N}{2}}_N}_{=-1} = -w_N^n
    \end{align}
  \end{mdframed}
folgt 
  \begin{mdframed}[style=exercise,font=\scriptsize]
    \begin{align}
        Y(n)&=\sum_{k=0}^{N/2-1} y(2k) w_{\frac{N}{2}}^{nk} + w_{\frac{N}{2}}^{n} \sum_{k=0}^{N/2-1} y(2k+1)w_{\frac{N}{2}}^{kn}\\
        Y(n+\frac{N}{2})&=\sum_{k=0}^{N/2-1} y(2k) w_{\frac{N}{2}}^{nk} - w_{\frac{N}{2}}^{n} \sum_{k=0}^{N/2-1} y(2k+1)w_{\frac{N}{2}}^{kn}
    \end{align}
  \end{mdframed}
  \begin{mdframed}[style=exercise]
    \begin{align}
        Y_{left}(n) &= Y_{even}(n) + w_{\frac{N}{2}}^{n} Y_{odd}(n)\\ 
        Y_{right}(n) &= Y_{even}(n) - w_{\frac{N}{2}}^{n} Y_{odd}(n)
    \end{align}
  \end{mdframed}
Die Komplexität ist $O(N\log_2(N))$:\\ 
Es gibt 2 $log_2(N)$ Splitting-Steps mit je $O(n)$
\newpage

%%%%%%%%%%%%%%%%%%%%%%%%%%%%%%%%%%%%%%%%%%%%%  Multirate %%%%%%%%%%%%%%%%%%%%%%%%%%%%%%%%%%%%%%%%%%%%%%%%%%%%%
\section{Multirate:}

\subsection{Änderung der Abtastfrequenz:}
\textbf{Wunsch: } Annpassung der Sample-Rate an das abzutastende Signal  
$\rightarrow$ Variable Abtastfrequenz mit gegebenen Samples
\subsubsection{Dezimieren:}
\textbf{Vorteile einer Reduzierten Taktzahl: }
\begin{itemize}
  \item geringerer Energieverbrauch 
  \item Reduzieren des Frequenzbandes (\grqq{}nur was interessiert\grqq{})
\end{itemize}

\textbf{ACHTUNG: } Anti-Aliasing-Filter sollte vorgeschalten werden
\begin{center}
  \includegraphics[width=.4\textwidth]{./img/Multirate_Dezimator.png}
\end{center} 
\begin{center}
  \includegraphics[width=.5\textwidth]{./img/Dezimator_Spektrum.png}
\end{center} 

\subsubsection{Interpolation:}

\begin{center}
  \includegraphics[width=.5\textwidth]{./img/Multirate_Interpolator.png}
\end{center} 

\begin{center}
  \includegraphics[width=.5\textwidth]{./img/Interpolator_Spektrum.png}
\end{center} 

\textbf{Filterung nach Interpolation: }\\

\begin{center}
  \includegraphics[width=.5\textwidth]{./img/Multirate_Interpolator_Filter.png}
\end{center}
\begin{center}
  \includegraphics[width=.5\textwidth]{./img/Multirate_Interpolator_Spektrum.png}
\end{center}

\subsubsection{Problem bei Interpolation oder Dezimierung: } 
\textbf{Problem:}
\begin{itemize}
  \item eingesetzten Filter haben harte Anforderungen
  \item Phasenlinearität verlangt nach komplexen FIR-Filtern
\end{itemize}

\textbf{Lösung:}
\begin{itemize}
  \item Dezimieren / Interpolieren in mehreren Stufen (einfachere Filter)
  \item Verwendung von Alternativlösungen für Filterung $\rightarrow$ z.B.: CIC-Filter
\end{itemize}

\subsection{Cascaded Integrator Comb (CIC):}
\subsubsection{Allgemeines :}
\begin{itemize}
  \item effiziente Architektur zum Filtern bei hohen Dezimations- / Interpolations-Raten
  \item \textbf{Verhalten: } Moving-Average-Filter
  \item benötigt keine Koeffizienten-Speicher
  \item vergleichbarer nichtrekursiver FIR-Filter würde 5 \\ ($z^{-6} \rightarrow D-1=6-1=5$) Addierer benötigen  
\end{itemize}

\begin{center}
  \includegraphics[width=.5\textwidth]{./img/Multirate_CIC.png}
\end{center}

\begin{center}
  \includegraphics[width=.5\textwidth]{./img/Multirate_CIC_Frequenzgang.png}
\end{center} 

\subsubsection{Verbesserungen: }
\begin{itemize}
  \item Kaskadierung von CIC-Filtern \\
  (Verbesserung der Filterwirkung)
  \item Anwendung der Noble-Äquivalenzen\\
   (Reduziert Speicherzellen-Anzahl)
\end{itemize}

\textbf{CIC-Dezimator (Noble): }
\begin{center}
  \includegraphics[width=.4\textwidth]{./img/CIC_Dezimator.png}
\end{center} 

\textbf{CIC-Interpolator (Noble): }
\begin{center}
  \includegraphics[width=.4\textwidth]{./img/CIC_Interpolator.png}
\end{center} 
 
%%%%%%%%%%%%%%%%%%%%%%%%%%%%%%%%%%%%%%%%%%%%%  FIR in FPGA %%%%%%%%%%%%%%%%%%%%%%%%%%%%%%%%%%%%%%%%%%%%%%%%%%%%%
\section{FIR in FPGA}
  \begin{center}
      \includegraphics[width=.35\textwidth]{./img/firfpga.png}
  \end{center}
Pro Takt müssen 1 Multiplizierer und N Addierer durchlaufen werden.\\
Die max. Taktfrequenz $f_{clk}$ ist
  \begin{mdframed}[style=exercise]
    \begin{align}
        f_{clk} = \frac{1}{T_{mul}+NT_{add}}
    \end{align}
  \end{mdframed}

\subsection{FIR - Pipelined} 
Jede Operationsstufe wird mit einem Pipelineregister gebuffert.\\
\textbf{Vorteil: } Massive Erhöhung der Taktfrequenz\\
\textbf{Nachteil: } zusätzliche Register, Latenz (Zeitl. Versatz (hier 4))
  \begin{center}
      \includegraphics[width=.35\textwidth]{./img/firpipeline.png}
  \end{center}
Die max. Taktfrequenz $f_{clk}$ annähernd durch $T_{mul}$ definiert:
  \begin{mdframed}[style=exercise]
    \begin{align}
        f_{clk} = \frac{1}{max(T_{mul},T_{add})} = \frac{1}{T_{mul}} 
    \end{align}
  \end{mdframed}

\subsection{FIR - transposed (DF1)}
Bevorzugte Implementierungsvariante
  \begin{center}
      \includegraphics[width=.35\textwidth]{./img/firtransposed.png}
  \end{center}
\textbf{Vorteil: } Pipeline-Register für Addierer werden eingspart, keine Latenz
  \begin{mdframed}[style=exercise]
    \begin{align}
        f_{clk}=\frac{1}{T_{mult}+T_{add}}
    \end{align}
  \end{mdframed}

\subsection{FIR - lineare Phase}
\textbf{Voraussetzung: } NST \grqq{}gespiegelt\grqq{} am Einheitskreis\\
\\\textbf{Vorteil: } Koeffizienten sind symmetrisch\\ $\rightarrow$ Anzahl an Multiplizierer kann halbiert werden
  \begin{center}
      \includegraphics[width=.35\textwidth]{./img/firlinear.png}
  \end{center}

\subsection{FIR - seriell}
Addierer und Multiplizierer sind teuere Ressourcen im FPGA.\\
Ein Controller Steuert einen MUX(Inputwerte) und RAM(Koeffs) und AKKU an.
  \begin{center}
      \includegraphics[width=.35\textwidth]{./img/firseriell.png}
  \end{center}
Die Maximale Taktfrequenz des FPGA $f_{clk}$ hängt von der Filterordnung $N$ 
und der Abtastfrequenz $f_A$ ab und muss min.
  \begin{mdframed}[style=exercise]
    \begin{align}
        f_{clk}= M \cdot f_A = (N-1)\cdot f_A
    \end{align}
  \end{mdframed}

\subsection{FIR - semi-parallel}
Bei der Seriellen Architektur ist die max. Filterordnung stark begrenzt. ($f_{max,FPGA}$ im 3st. MHz-Bereich )
Für höhere Filterordnung kann die semi-parallele Architektur verwendet werden.
  \begin{center}
      \includegraphics[width=.35\textwidth]{./img/firsemi.png}
  \end{center}
 \textbf{K:} Sektionen \textbf{M:} Multiplizierer \\
Die max. Taktfrequenz des FPGA muss jetzt nur noch ein K-tel so hoch sein:
  \begin{mdframed}[style=exercise]
    \begin{align}
        f_{clk}= \frac{M}{K}\cdot f_A = \frac{N-1}{K} \cdot f_A
    \end{align}
  \end{mdframed}

\section{NCO}
\begin{itemize}
  \item NCO = \grqq{}Numerically Controlled Oscillator\grqq{}
  \item Phasenakkumolator, der Jeden Takt um ein Phaseninkrement $\mu$ erhöht wird
  \item Ausgang des Counters wird mit Look-Up-Table (LUT) in Signalform (sin,cos,sägezahn) umgewandelt \\(\textbf{PAC:} Phase Amplitude Converter)
  \item LUT ist mit $N = 2^n$ 12-bit breiten Werten gefüllt 
\end{itemize}

\begin{center}
  \includegraphics[width=.5\textwidth]{./img/NCO_schema.png}
\end{center}

\begin{center}
    \includegraphics[width=.5\textwidth]{./img/nco.png}
\end{center}

  \begin{mdframed}[style=exercise]
    \begin{align}
        \mu = N \frac{f_d}{f_s}  
    \end{align}
  \end{mdframed}

%%%%%%%%%%%%%%%%%%%%%%%%%%%%%%%%%%%%%%%%%%%%% Nützliche Code-Stückchen %%%%%%%%%%%%%%%%%%%%%%%%%%%%%%%%%%%%%%%%%%%%%%%%%%%%%
\section{Quantisierung}
Ein ADC gibt Ganzzahlwerte von 0 bis N zurück. Wir müssen diese Ganzzahl auf die Referenzspannung $U_{Ref}$ beziehen.
Der exakte Spannungswert $U(k) = U_{Ref}\cdot \frac{k}{N}$. \\
Dieses Ergebnis wird in gewisser Weise durch eines der Verfahren quantisiert:
\begin{itemize}
    \item Runden (Bevorzugte Variante)
    \item Abschneiden (Wert fällt auf nächstiefere)
    \item Betragsabschneiden (Wert fällt Richtung 0)
\end{itemize}

\subsection{Quantisierung mit Runden}
\begin{center}
    \includegraphics[width=.45\textwidth]{./img/rundung.png}
\end{center}
Es ensteht ein Fehler beim Quantisieren $e(n)$ 
  \begin{mdframed}[style=exercise]
    \begin{align}
        e(n) &= Q\{ x(n) \} - x(n)\\
        -\frac{\Delta}{2} & < e(n) \leq \frac{\Delta}{2}
    \end{align}
  \end{mdframed}

\begin{center}
    \includegraphics[width=.45\textwidth]{./img/rangefullscale.png}
\end{center}
Der Volle Eingangsbereich $R_{FS}$ teilt sich auf $2^B-1$ Teile der Länge $\Delta$ auf.
  \begin{mdframed}[style=exercise]
    \begin{align}
        R_{FS} = \Delta \cdot 2^{B}
    \end{align}
  \end{mdframed}

\subsection{Modelierung des Quantisierungsfehlers}
Versuch den Quantisierungsfehlers zahlenmäßig zu erfassen.
$e(n)$ kann durch unkorreliertes, mittelwertfreies, gleichverteiltes, weißes Rauschen modeliert werden.
Dieses Modell ist zulässig wenn
\begin{itemize}
    \item Quantisierungsstufe $\Delta$ klein in Vergleich zur Signalamplitude
    \item Das Signal soll einige Q-Stufen zwischen zwei Abtastwerten überqueren
    \item Kein Überlauf und keine Saturation
    \item x(n) nicht periodisch mit Vielfaches von $f_A$
    \item Rauschen muss vom Signal kommen
\end{itemize}

\subsection{Quantisierungsrauschleistung}
  \begin{mdframed}[style=exercise]
    \begin{align}
        E_e = \sigma_e^2 = \frac{\Delta^2}{12} = \frac{R_{FS}^2}{12\cdot 2^{2B}} \\
    \end{align}
  \end{mdframed}
Bsp:
\begin{center}
    \includegraphics[width=.4\textwidth]{./img/sqnr_eq.png}
\end{center}

\subsection{Signal-to-Quantization Noise Ratio SQNR}
  \begin{mdframed}[style=exercise]
    \begin{align}
        S_{max} = \frac{A_{max}^2}{2} \approx \frac{R_{FS}}{8} = 2^{2B-3}\cdot \Delta^2
    \end{align}
  \end{mdframed}
mit Quantisierungsrauschleistung $\sigma_e^2$ ergiebt sich
  \begin{mdframed}[style=exercise]
    \begin{align}
        SQNR_{max} = \frac{S_{max}}{\sigma^2_e} = \frac{2^{2B-3}\cdot \Delta^2}{ \frac{\Delta^2}{12} } = (6,02B+1,76)dB
    \end{align}
  \end{mdframed}
Es ergiebt sich eine Verschlechterung von $\frac{6dB}{Bit}$

\subsection{Addieren}
Pro Addition verlängert sich das Ergebnis um 1 Bit
\begin{enumerate}
    \item im FPGA die Wortbreite nach jedem Addierer erhöhen
    \item auf dem DSP wählt man die Wortlänge (16 / 32 Bits)
    \item Wenn zu lang, herunter skalieren (LSBs abschneiden)
    \item Bedingt herunter skalieren und die Anzahl der Skalierungen merken („block floating point“)
    \item Nichts tun, wenn die Signale klein genug sind
\end{enumerate}

\subsection{Multiplizieren}
Beim Multiplizieren ist die Ergebnislänge die Summe der Längen
der Faktoren.\\
Konsequenzen:
\begin{enumerate}
    \item Nach jeder Multiplikation ist eine Skalierung wegen der schnell wachsender Länge denkbar
    \item In FPGA ist es ratsam, die Längen der fest implementierten Multiplizierer auszunutzen (z.B. 18x18)
    \item Wenn sinnvoll, Multiplikation mit Konstanten als CSD implementieren (dann nur Additionen)
\end{enumerate}

\subsection{Grenzzyklus}
Schwingungen die sich durch die Quantisierung einstellen.
\begin{center}
    \includegraphics[width=.4\textwidth]{./img/grenzzyklus.png}
\end{center}

\section{Nützliche Code-Ausschnitte:}
\textbf{Maximalwert-Ermittlung z.B. für Sepktralauswertung:}
\begin{minted}[fontsize=\scriptsize]{c}
max_value = 0 ; //Startwert des Vergleichs
max_idx = 2 ; //Startindex des Vergleichs
for(int i = 2; i<(N-1); i++){ 
  if( y[i] > max_value){
    max_value = y[i];
    max_idx = i;
  }
}
\end{minted}


\end{document}
