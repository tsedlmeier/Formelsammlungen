\documentclass[10pt,a4paper]{article}
\usepackage[utf8]{inputenc}
\usepackage[ngerman]{babel}
\usepackage[T1]{fontenc}
\usepackage{amsmath}
\usepackage{amsfonts}
\usepackage{amssymb}
\usepackage{graphicx}
\usepackage{lmodern}
\usepackage{physics}
\usepackage[left=1cm,right=1cm,top=1.5cm,bottom=1.2cm]{geometry}
\usepackage{siunitx}
\usepackage{fancyhdr}
\usepackage{enumerate}
\usepackage{mhchem}
\usepackage{mathtools}
\usepackage{graphicx}
\usepackage{float}
\usepackage[table]{xcolor}
\usepackage{mdframed}
\usepackage{csquotes}
\usepackage{trfsigns}
\usepackage{capt-of}
\usepackage{adjustbox}
\usepackage{verbatim}

\sisetup{locale=DE}
\sisetup{per-mode = symbol-or-fraction}
\sisetup{separate-uncertainty=true}
\DeclareSIUnit\year{a}
\DeclareSIUnit\clight{c}
\mdfdefinestyle{exercise}{
	backgroundcolor=black!10,roundcorner=8pt,hidealllines=true,nobreak
}

\begin{document}
\twocolumn
\pagestyle{fancy}
% \lhead{DSV Formelsammlung, Stand {\input{\string"| date + " %Y-%d-%m" \string"}}}
\lhead{Formelsammlung Mathe, \today}
\rhead{Sedlmeier, Toni}
\section{Begriffe - Linearität und Zeitinvarianz}
%%%%%%%%%%%%%%%%%%%%%%%%%%%%%%%%%%%%% Elementare System-Eigenschaften %%%%%%%%%%%%%%%%%%%%%%%%%%%%%%%%%%%%%%%%%%%%
  \subsection{Linear/Nichtlinear}
  Definition \textbf{Lineares} System:
  \begin{mdframed}[style=exercise]
    \begin{align}
        u = \alpha_1 u_1(t)+ \alpha_2 u_2(t)\rightarrow y = \alpha_1 y_1(t)+ \alpha_2 y_2(t) \\  
    \end{align}
  \end{mdframed}
  Nichtlinearitäten: 
  \begin{itemize}
    \item u(t) ist in nichtlin. Funktion $f$ (z.B $sin$, $sqrt$, $e$) versteckt
    \item Addition mit Konstante $k$
    \item Kennlinien (Begrenzer, Vorzeichenwechsel, Quantisierung)
  \end{itemize}
  \begin{mdframed}[style=exercise]
    \begin{align}
        u(t)\rightarrow y(t) &= f\{u(t)\} \\
        u(t)\rightarrow y(t) &= u(t) + k
    \end{align}
  \end{mdframed}

  \subsection{Zeitvariant/Zeitinvariant}
  Definition \textbf{Zeitinvariantes} System:
  \begin{mdframed}[style=exercise]
    \begin{align}
        u(t-\tau) \rightarrow y(t-\tau)
    \end{align}
  \end{mdframed}
  Zeitvarianz: 
  \begin{itemize}
    \item Multiplikation mit $f(t)$
    \item Zeitverschiebung von \textbf{nur} $u(t)$ bzw. $y(t)$
  \end{itemize}
  \begin{mdframed}[style=exercise]
    \begin{align}
        u(t) &\rightarrow y(t) = f(t)u(t)\\
        u(t) &\rightarrow y(t) = u(t-t_0)\\
        u(t) &\rightarrow y(t-t_0) = u(t)
    \end{align}
  \end{mdframed}

\subsection{$DGL^n$ $\rightarrow$ $DGL-System^1$}
Eine DGL n-ter Ordnung kann in ein DGL-System n-ter Ordnung umgewandelt werden:
  \begin{mdframed}[style=exercise]
    \begin{align}
        a_n y^{(n)}(t) + ... + a_2 y''(t) + a_1 y'(t)+ a_0 y(t)= b_0u(t)
    \end{align}
  \end{mdframed}
Lösungsansatz für n=4: 
  \begin{mdframed}[style=exercise]
    \begin{align}
        x_1(t) &= y(t) & \indent      x_1^{'}(t) &= x_2(t)\\
        x_2(t) &= y^{'}(t) &\indent   x_2^{'}(t) &= x_3(t)\\
        x_3(t) &= y^{''}(t) &\indent  x_3^{'}(t) &= x_4(t)\\
        x_4(t) &= y^{(3)}(t) &\indent 
    \end{align}
  \end{mdframed}
  \begin{mdframed}[style=exercise]
    \begin{align}
        x_4^{'}(t) &= \frac{1}{a_4}[b_0 u(t) -a_3x_4(t) - a_2x_3(t)-a_1x_2(t)-a_0x_1(t)]
    \end{align}
  \end{mdframed}
Zustandsraumdarstellung:
  \begin{mdframed}[style=exercise]
    \begin{align}
        \begin{pmatrix}
        x_1^{'}(t) \\
        x_2^{'}(t) \\
        x_3^{'}(t) \\
        x_4^{'}(t) \\
        \end{pmatrix} 
        =
        \begin{pmatrix}
            0 & 1 & 0 & 0 \\
            0 & 0 & 1 & 0 \\
            0 & 0 & 0 & 1 \\
            -\frac{a_0}{a_4} & -\frac{a_1}{a_4} & -\frac{a_2}{a_4} &  -\frac{a_3}{a_4}\\
        \end{pmatrix}
        \begin{pmatrix}
        x_1(t) \\
        x_2(t) \\
        x_3(t) \\
        x_4(t) \\
        \end{pmatrix} 
        +
        \begin{pmatrix}
        0 \\
        0 \\
        0 \\
        \frac{b_0}{a_4}u(t) \\
        \end{pmatrix} 
    \end{align}
  \end{mdframed}


\section{Matrizenrechnung}
\subsection{Eigenwerte und Eigenvektoren}
Eine Zahl $\lambda \in \mathbb{C}$ heißt Eigenwert von Matrix $A$($n$x$n$),
falls die Multiplikation von $A$ mit Vektor $v$ $\in \mathbb{C}^n$ in eine Skalarmultiplikation
mit $\lambda$ zusammenfällt. Dieser Vektor $v$ wird dann Eigenvektor genannt.
  \begin{mdframed}[style=exercise]
    \begin{align}
        Av = \lambda v
    \end{align}
  \end{mdframed}

\subsection{Eigenwertprobleme}
EW = Nullstellen des char. Polynoms $p_A(\lambda)$.\\
  \begin{mdframed}[style=exercise]
    \begin{align}
        det(A-\lambda E) \Longleftrightarrow 
        \begin{vmatrix}
             a_{11}-\lambda & a_{12} & ...  & a_{1n} \\
             a_{21}& a_{22}-\lambda & ... & a_{2n} \\
              ... & ...  & ...  & a_{3n} \\
             a_{n1} & a_{n2} & ... &  a_{nn}-\lambda\\
        \end{vmatrix} = 0
    \end{align}
  \end{mdframed}
  \begin{mdframed}[style=exercise]
    \begin{align}
        p_A(\lambda) &= a_n \lambda^n +...+ a_1 \lambda + a_0\\
        p_A(\lambda) &= (\lambda - \lambda_1)(\lambda - \lambda_2)...(\lambda - \lambda_n)
    \end{align}
  \end{mdframed}

  \begin{itemize}
    \item $p_A(\lambda)$ hat $n$ Nulltellen
    \item VFH der NS $\lambda_i$ = \textbf{algebraische} VFH des EW $a(\lambda_i)$ 
    \item $\sum_i a(\lambda_i) = n$ 
    \item falls A \textbf{symmetrisch} $\rightarrow$ alle EW sind \textbf{reel}
  \end{itemize}
Für Eigenvektoren einfach die EW auf der Hauptdiagonalen abziehen
  \begin{mdframed}[style=exercise]
    \begin{align}
        (A-\lambda E)v = 0
    \end{align}
  \end{mdframed}
Bestimmung des i-ten EV $v_i$ mit zugehörigem EW $\lambda_i$ \\
Bsp für n = 2: (v bestimmen durch Hinschauen)
  \begin{mdframed}[style=exercise]
    \begin{align}
        \lambda_i : 
        \begin{pmatrix}
            a_{11}-\lambda_i & a_{12} \\
            a_{21} & a_{22}-\lambda_i \\
        \end{pmatrix} v = 0
    \end{align}
  \end{mdframed}

  \begin{itemize}
    \item \textbf{geometrische} VFH $g(\lambda_i)$ = Anzahl lin. unabh. EV $v$
  \end{itemize}

  \subsection{Geometrische Lage von EW}
  EW-Bestimmung teilweise aufwendig. Aussage über Stabilität von DGL-Systemen anhand der Lage der EW der Systemmatrix A 
  (z.B alle EW links in komplexer Halbebene).\\
  Kriterien:
  \begin{itemize}
    \item Gerschgorin-Kreise 
    \item Routh-Hurwitz-Kriterium
  \end{itemize}

  \subsubsection{Gerschgorin-Kreise}
Aussage über geometrische Lage der EW einer Matrix.
Ablesen von Systemeigenschaften ohne tatsächliche Berechnung der EW.\\
Vorgehen:
  \begin{itemize}
    \item Für jede Zeile ein Kreis
    \item Diagonalelement ist der Mittelpunkt
    \item Summe der Beträge der Restelemente der Radius
  \end{itemize}
Bsp für n = 3
  \begin{mdframed}[style=exercise]
    \begin{align}
        \begin{pmatrix}
             a_{11} & a_{12} & a_{13} \\
             a_{21} & a_{22} & a_{23} \\
             a_{31} & a_{32} & a_{33} 
        \end{pmatrix}
    \end{align}
  \end{mdframed}
  dann
  \begin{mdframed}[style=exercise]
    \begin{align}
        K1: MP=(a_{11}/0), R=|a_{12}+a_{13}|\\
        K1: MP=(a_{22}/0), R=|a_{21}+a_{23}|\\
        K1: MP=(a_{33}/0), R=|a_{31}+a_{32}|
    \end{align}
  \end{mdframed}
  \begin{itemize}
    \item Disjunkte Kreise enthalten genau einen EW
    \item Vereinigung von $m$ disjunkten Kreisen enthält $m$ EW
  \end{itemize}

  \subsection{Routh-Hurwitz-Kriterium}
  Aussage EW in \textbf{linker Halbebene} (negativer Realteil)
  über Koeffizienten des charakteristischen Polynoms $p_A(\lambda)$

  \begin{mdframed}[style=exercise]
    \begin{align}
        p_A(\lambda) &= a_n \lambda^n +...+ a_1 \lambda + a_0\\
        p_A(\lambda) &= (\lambda - \lambda_1)(\lambda - \lambda_2)...(\lambda - \lambda_n)
    \end{align}
  \end{mdframed}
\textbf{Aussagen}: (gilt in beide Richtungen)
\begin{enumerate}[{(1)}]
    \item Alle Nulltellen $\lambda_n$ besitzen \textbf{negativen Realteil}
    \begin{center} $\Updownarrow$\end{center}
    \item Alle Koeffizienten $a_n$ sind \textbf{positiv} und alle \textbf{HUD} der folgenden Matrix $H$ sind \textbf{positiv}
\end{enumerate}

  \begin{mdframed}[style=exercise]
    \begin{align} H =
        \begin{pmatrix}
            a_{n-1} & a_{n-3} & a_{n-5} & ...   & a_{n-(2n-1)} \\
             a_{n}  & a_{n-2} & a_{n-4} & ...   & a_{n-(2n-2)}\\
              0     & a_{n-1} & a_{n-3} & ...   & a_{n-(2n-2)}\\
             ...    &  ...    &  ...    & ...   &  ... \\
              0     &    0    &    0    & ...   &  a_{0}
        \end{pmatrix}
    \end{align}
  \end{mdframed}

  \subsection{Diagonalisierbarkeit und Jordan-\\Normalform}
  Eine Matrix von Typ ($n$x$n$) heißt \textbf{diagonalisierbar}, 
  wenn es eine reguläre Matrix $T$ gibt, sodass 
  \begin{mdframed}[style=exercise]
    \begin{align}
        T^{-1} \cdot A \cdot T = D 
    \end{align}
  \end{mdframed}
  die Diagonalmatrix $D$ liefert.

\subsubsection{1. Fall: A Diagonalisierbar}
\begin{enumerate}[{(1)}]
    \item geometrische VF = algebraische VF für alle EW
  \begin{mdframed}[style=exercise]
    \begin{align}
        g(\lambda_i) = a(\lambda_i) \ \ i \in [1..n]
    \end{align}
  \end{mdframed}
    \item Es existieren $n$ linear unabhängige EW, die die Spalten der
        der Diagonalisierungsmatrix $V$ liefern
  \begin{mdframed}[style=exercise]
    \begin{align}
        V^{-1} \cdot A \cdot V &= D \\
        V =
        \begin{pmatrix}
            v_1 & v_2 & v_3 & ... & v_n
        \end{pmatrix}\ ;\ 
        D &=
        \begin{pmatrix}
            \lambda_1& 0    & 0  & 0 \\
             0  & \lambda_2  & 0 & 0 \\
             0 & 0 &\lambda_3& ... \\
             0 & 0& ... & \lambda_n 
        \end{pmatrix}
    \end{align}
  \end{mdframed}
\end{enumerate}

\subsubsection{2. Fall: A Nicht-Diagonalisierbar}
\begin{enumerate}[{(1)}]
    \item geometrische VF < algebraische VF für min. 1 EW
  \begin{mdframed}[style=exercise]
    \begin{align}
        g(\lambda_i) < a(\lambda_i) \ \ i \in [1..n]
    \end{align}
  \end{mdframed}
    \item Es können fehlende linear unabhängige EW, durch HV $\textcolor{red}{v_k}$ ersetzt werden
  \begin{mdframed}[style=exercise]
    \begin{align}
        V^{-1} \cdot A \cdot V &= J \\
        V =
        \begin{pmatrix}
            v_1 & v_2 & \textcolor{red}{v_k} &... & v_n
        \end{pmatrix}\ ;\ 
        D &=
        \begin{pmatrix}
             \lambda_1& 0  & 0& 0 \\
             0 & \lambda_2 & 1 & 0 \\
             0 & ...       & \textcolor{red}{v_k} & ... \\
             0 & 0 & ... & \lambda_n \\
        \end{pmatrix} 
    \end{align}
  \end{mdframed}
\end{enumerate}

\subsubsection{Darstellungsformen der Jordan-Normalform}
Jede Matrix vom Typ ($n$x$n$) kann in eine Jordan-Normalform $J$ gebracht werden.
\\Es gilt: 
\begin{itemize}
    \item i-ter Jordan-Block $J_i$ besteht aus EV/HV 
    \item geo. VFH (= Anzahl EV) bestimmt größe des Jordan-Blocks 
\end{itemize}
  \begin{mdframed}[style=exercise]
    \begin{align}
        J &=
        \begin{pmatrix}
             J_1 & 0  & 0 & 0 \\
             0 & J_2  & 0 & 0 \\
             0 & 0  & ... & ... \\
             0 & 0    & ... & J_n \\
        \end{pmatrix} 
    \end{align}
  \end{mdframed}
Jordan-Normalform kann durch Diagonalmatrix D und Nilpotente Matrix N.
N kann durch Potenzieren zur Nullmatrix zerfallen ($N^m = 0$)

  \begin{mdframed}[style=exercise]
    \begin{align}
\underbrace{ 
        \begin{pmatrix}
             \lambda_i& 1  & 0& 0 \\
             0 & \lambda_i & 1 & 0 \\
             0 & 0 & \lambda_i & ... \\
             0 & 0 & ... & \lambda_i \\
        \end{pmatrix} 
        }_{J_i} = 
\underbrace{ 
        \begin{pmatrix}
             \lambda_i& 0  & 0& 0 \\
             0 & \lambda_i & 0 & 0 \\
             0 & 0 & \lambda_i & ... \\
             0 & 0 & ... & \lambda_i \\
        \end{pmatrix} 
        }_{D_i} +
\underbrace{ 
        \begin{pmatrix}
             0 & 1  & 0& 0 \\
             0 & 0 & 1 & 0 \\
             0 & 0 & 0 & ... \\
             0 & 0 & ... & 0 \\
        \end{pmatrix} 
        }_{N_i}
    \end{align}
  \end{mdframed}
Mit $J_i$ = i-ter Jordan-Block. \\
\begin{itemize}
    \item Anzahl der Jordan-Blöcke = geometrische VFH
\end{itemize}

Bsp für n = 3:
\begin{enumerate}[(1)]
    \item \textbf{1.Fall: geometrische VHF = 3}\\
        $\rightarrow J = (v_1 \ v_2 \ v_3) = D$

  \begin{mdframed}[style=exercise]
    \begin{align}
    J = \begin{pmatrix}
        \textcolor{red}{\lambda_1}  & | & 0  & | & 0 \\
            --          & | &--              & | & -- \\
             0          & | & \textcolor{blue}{\lambda_2} & | & 0 \\
            --          & | & --             & | &  --\\
             0          & | & 0              & | & \textcolor{green}{\lambda_3}
        \end{pmatrix} = 
        \begin{pmatrix}
            \textcolor{red}{J_1} & 0 & 0\\
            0 & \textcolor{blue}{J_1} & 0\\
            0 & 0 & \textcolor{green}{J_3}
        \end{pmatrix}
    \end{align}
  \end{mdframed}
Jordan-Blöcke $J_i$ sind jeweil (1x1)
    \item \textbf{2.Fall: geometrische VHF = 2}\\
        $\rightarrow J = (v_1 \ t_1 \ v_2)$

  \begin{mdframed}[style=exercise]
    \begin{align}
    J = \begin{pmatrix}
        \textcolor{red}{\lambda_1}  & 1  & | & 0 \\
             0            & \textcolor{red}{\lambda_2} & | & 0 \\
            --            & --             & | &  --\\
             0            & 0              & | & \textcolor{blue}{\lambda_3}
        \end{pmatrix} = 
        \begin{pmatrix}
            \textcolor{red}{J_1} & 0 \\
            0 & \textcolor{blue}{J_2}\\
        \end{pmatrix}
    \end{align}
  \end{mdframed}
    Jordan-Block $J_1$ ist (2x2), $J_2$ ist (1x1)

    \item \textbf{3.Fall: geometrische VHF = 1}\\
        $\rightarrow J = (v_1 \ t_1 \ t_2)$

  \begin{mdframed}[style=exercise]
    \begin{align}
    J = \begin{pmatrix}
        \textcolor{red}{\lambda_1}  & 1  &  0 \\
             0            & \textcolor{red}{\lambda_2} & 1 \\
             0            & 0              & \textcolor{red}{\lambda_3}
        \end{pmatrix} = 
        \begin{pmatrix}
            \textcolor{red}{J_1} \\
        \end{pmatrix}
    \end{align}
  \end{mdframed}
\end{enumerate}
Jordan-Block ist gesammte Matrix (3x3)

\subsubsection{Hauptvektoren}
Ein Hauptvektoren $t \in \mathbb{C}^n$ ist ein Vektor mit der Eigenschaft 
  \begin{mdframed}[style=exercise]
    \begin{align}
        &(A-\lambda E)^j \cdot t = 0 \\ 
        &(A-\lambda E)^{j-1} \cdot t \neq 0
    \end{align}
  \end{mdframed}

  \begin{mdframed}[style=exercise]
    \begin{align}
        &(A-\lambda E)\cdot t^{(1)} = 0  \ \ \ \ \ (HV0 = EV)\\
        &(A-\lambda E)\cdot t^{(2)} = t^{(1)} \neq 0\ \ \ \ \ (HV1)\\
        &(A-\lambda E)\cdot t^{(3)} = t^{(2)}\neq 0\ \ \ \ \ (HV2) 
    \end{align}
  \end{mdframed}

  \subsection{Matrix-Exponentialfunktion}
  Lösungsansatz für lin. homog. DGL-Systeme
  \begin{mdframed}[style=exercise]
    \begin{align}
        x(t) = c \cdot e^{At}
    \end{align}
  \end{mdframed}
  Mit Potenzreihe:
  \begin{mdframed}[style=exercise]
    \begin{align}
        e^{At} = E + t\ A+ \frac{t^2}{2!}A^2+ \frac{t^3}{3!}A^3 ...
    \end{align}
  \end{mdframed}
\textbf{Rechenregeln:}
\begin{enumerate}
    \item Assoziativität
  \begin{mdframed}[style=exercise]
    \begin{align}
        e^{(A+B)t} &= e^{tA} \cdot e^{tB}\\
        e^{A(t+s)} &= e^{tA} \cdot e^{sA}
    \end{align}
  \end{mdframed}

    \item Inverse
  \begin{mdframed}[style=exercise]
    \begin{align}
        (e^{tA})^{-1} = e^{-tA}
    \end{align}
  \end{mdframed}

    \item Ableitung
  \begin{mdframed}[style=exercise]
    \begin{align}
        \frac{d(e^{tA})}{dt} = A \cdot e^{tA} =e^{tA} \cdot A
    \end{align}
  \end{mdframed}

    \item Eigenvektor $v$ zum Eigenwert $\lambda$
  \begin{mdframed}[style=exercise]
    \begin{align}
        e^{At} \cdot v = e^{\lambda t} \cdot v
    \end{align}
  \end{mdframed}

    \item Hauptvektor $v$ zum Eigenwert $\lambda$
  \begin{mdframed}[style=exercise]
    \begin{align}
        e^{At} \cdot v = e^{\lambda t} (v+\frac{t}{1!}(A-\lambda E )+\frac{t^2}{2!}(A-\lambda E )^2 + ... )
    \end{align}
  \end{mdframed}

    \item Blockdiagonalmatrix $A$ 
  \begin{mdframed}[style=exercise]
    \begin{align}
     A=   \begin{pmatrix}
             A_1& 0  & 0\\
             0 & A_2 & ... \\
             0 & ... & A_n \\
        \end{pmatrix} \rightarrow 
        e^{tA} =   \begin{pmatrix}
            e^{tA_1}& 0  & 0\\
             0 & e^{tA_2} & ... \\
             0 & ... & e^{tA_n} \\
        \end{pmatrix}
    \end{align}
  \end{mdframed}

\end{enumerate}

\subsubsection{Berechnung $e^{tA}$}
\textbf{1.Fall:} A Diagonalisierbar
  \begin{mdframed}[style=exercise]
    \begin{align}
        e^{tA} = V \cdot e^{tD} \cdot V^{-1} \\
        V =
        \begin{pmatrix}
            v_1 & v_2 & v_3 & ... & v_n
        \end{pmatrix}; 
        e^{tD} &=
        \begin{pmatrix}
            e^{t\lambda_1}& 0    & 0  & 0 \\
             0  & e^{t\lambda_2}  & 0 & 0 \\
             0 & 0 & e^{t\lambda_3}& ... \\
             0 & 0& ... & e^{t\lambda_n} 
        \end{pmatrix}
    \end{align}
  \end{mdframed}
\textbf{2.Fall:} A Nicht-Diagonalisierbar
\begin{enumerate}[(1)]
    \item Berechnung des i-ten Jordan-Block\\
        Bsp für n = 3
  \begin{mdframed}[style=exercise]
    \begin{align}
        e^{tJ_i} = e^{\lambda_i t N} \ \ \ \ ; \ mit \ \ 
        N = \begin{pmatrix}
            1 & t & \frac{t^2}{2!} \\
            0 & 1 &  t \\
            0 & 0 &  1 \\
        \end{pmatrix}
    \end{align}
  \end{mdframed}

    \item Jordan-Block zusammenfügen\\
        Bsp für n = 3
  \begin{mdframed}[style=exercise]
    \begin{align}
        e^{tJ} =
        \begin{pmatrix}
            e^{tJ_1} & 0 & 0 & 0 \\
            0 & e^{tJ_2} & ... & 0 \\
            0 &  ...  & ... &  ... \\
            0 & 0 & ... & e^{tJ_n} \\
        \end{pmatrix}
    \end{align}
  \end{mdframed}

    \item Ähnlichkeitstrafo ausführen\\
        Bsp für n = 3
  \begin{mdframed}[style=exercise]
    \begin{align}
        e^{tA} = T \cdot e^{tJ} \cdot T^{-1} \\
    \end{align}
  \end{mdframed}
\end{enumerate}


\section{Lösung DGL-Systeme}
\subsection{Lineare zeitvariable DGL-Systeme}
  \begin{mdframed}[style=exercise]
    \begin{align}
        \frac{dx(t)}{dt} = A(t)x(t) + B(t)u(t)
    \end{align}
  \end{mdframed}
Lösungsansatz der homogenen Lösung $x_H$
  \begin{mdframed}[style=exercise]
    \begin{align}
        x_H(t) = \Phi(t;t_0)x_0
    \end{align}
  \end{mdframed}
$\Phi(t;t_0)$ = Transitionsmatrix.\\
Berechnung von $\Phi(t;t_0)$ über Fundamentalmatrix $X(t)$
  \begin{mdframed}[style=exercise]
    \begin{align}
        \Phi(t;t_0) &= X(t) \cdot (X(t_0))^{-1} \\
        X(t) &= (x_1(t)\  x_2(t)\ ... \ x_n(t)) 
    \end{align}
  \end{mdframed}
Spezielle Lösung $x_S$ mittels Variation der Konstanten

  \begin{mdframed}[style=exercise]
    \begin{align}
        x_s(t) = \Phi(t;t_0)k(t)
    \end{align}
  \end{mdframed}

\subsection{Lineare zeitinvariante DGL-Systeme}
  \begin{mdframed}[style=exercise]
    \begin{align}
        \frac{dx(t)}{dt} = Ax(t) + Bu(t)
    \end{align}
  \end{mdframed}
Für LTI-DGL-Systeme gibt es 2 Lösungsvarianten.
\begin{enumerate}
    \item Matrix-Exponentialfunktion (geht immer)
    \item Lsg mittels EW/EV (nur falls A Diagonalisierbar)
\end{enumerate}
\subsubsection{Lsg mit Matrix-Exponentialfunktion}

\begin{enumerate}[(1)]
    \item Homogene Lsg $x_H$ mit Anfangsbed. $t_0$
  \begin{mdframed}[style=exercise]
    \begin{align}
        x_H(t) &= e^{tA} \cdot C
    \end{align}
  \end{mdframed}
  mit
  \begin{mdframed}[style=exercise]
    \begin{align}
        e^{tA} &= V \cdot e^{tD} \cdot V^{-1} \rightarrow A \ diagonalisierbar \\
        e^{tA} &= T \cdot e^{tJ} \cdot T^{-1} \rightarrow A \ nicht\ diagonalisierbar
    \end{align}
  \end{mdframed}

    \item Spezielle Lsg $x_S$ mittels V.d.K 
  \begin{mdframed}[style=exercise]
    \begin{align}
        x_S(t) = \displaystyle\int_{0}^{t} e^{(t-\tau)A} \cdot B u(t) d\tau
    \end{align}
  \end{mdframed}
    Alternativ: A.v.T.d.r.S (s. Papula)

\item Superlösung $x$ mit Anfangsbed. $x_0$
  \begin{mdframed}[style=exercise]
    \begin{align}
        x(t) = x_H(t) + x_S(t)
    \end{align}
  \end{mdframed}
  \begin{itemize}
      \item $x_0 = 
        \begin{pmatrix}
            x_{0,1} \\
            x_{0,2} \\
        \end{pmatrix}$ 
          in $x(t)$ einsetzen und nach $C = 
        \begin{pmatrix}
            C_1 \\
            C_2
        \end{pmatrix}$ 
          auflösen
  \end{itemize}
\end{enumerate}

\subsubsection{Lsg mit EW/EV (nur falls A Diagonalisierbar)}
\begin{enumerate}[(1)]
    \item Homogene Lsg $x_H$ mit Anfangsbed. $t_0$
  \begin{mdframed}[style=exercise]
    \begin{align}
        x_H(t) &= C_1\cdot e^{\lambda_1 t}v_1 + C_2 \cdot e^{\lambda_2 t}v_2 +... \\ \\
        x_H(t) &= V\cdot e^{tD}c = V
        \begin{pmatrix}
            e^{\lambda_1 t_1} & 0 & 0 & 0 \\
            0 & e^{\lambda_2 t_2}& ... & 0 \\
            0 &  ...  & ... &  ... \\
            0 & 0 & ... & e^{\lambda_n t_n}\\
        \end{pmatrix} c
    \end{align}
  \end{mdframed}

    \item Spezielle Lsg $x_S$ mittels V.d.K 
  \begin{mdframed}[style=exercise]
    \begin{align}
        x_S(t) = \displaystyle\int_{0}^{t} e^{(t-\tau)A} \cdot B u(t) d\tau
    \end{align}
  \end{mdframed}
    Alternativ: A.v.T.d.r.S (s. Papula)

\item Superlösung $x$ mit Anfangsbed. $x_0$
  \begin{mdframed}[style=exercise]
    \begin{align}
        x(t) = x_H(t) + x_S(t)
    \end{align}
  \end{mdframed}
  \begin{itemize}
      \item $x_0 = 
        \begin{pmatrix}
            x_{0,1} \\
            x_{0,2} \\
        \end{pmatrix}$ 
          in $x(t)$ einsetzen und nach $C = 
        \begin{pmatrix}
            C_1 \\
            C_2
        \end{pmatrix}$ 
          auflösen
  \end{itemize}
\end{enumerate}

\subsection{Ruhelagen Linearisierung und Stabilität}

\subsubsection{Ruhelagen}
Bestimmung Ruhelagen:
\begin{enumerate}
    \item Setzte $u(t) = u_R (const.)$
    \item Setzte $\frac{dx(t)}{dt} = 0$
\end{enumerate}

  \begin{mdframed}[style=exercise]
    \begin{align}
        f(x_R, u_R) = 0
    \end{align}
  \end{mdframed}

\subsubsection{Linearisierung um Ruhelage}
Wurden Ruhelagen gefunden so kann das DGL-System in der
Nähe linearisiert werden.
Bsp für n = 2
\begin{enumerate}
    \item 1. Gleichung nach $x_1$, $x_2$, $u$ ableiten
    \item 2. Gleichung nach $x_1$, $x_2$, $u$ ableiten
    \item Jacobi-Matrix $J_f$ aufstellen 
    \item Ruhelagen $x_R$ einsetzen
\end{enumerate}
Jacobi-Matrix $J_f$ mit $f_n$ n-te Gleichung und $x_n$ n-ter Zustand:
  \begin{mdframed}[style=exercise]
    \begin{align}
        J_f = 
        \begin{pmatrix}
            \frac{\partial f_1}{\partial x_1} & \frac{\partial f_1}{\partial x_2} \\ \\
            \frac{\partial f_2}{\partial x_1} & \frac{\partial f_2}{\partial x_2} 
        \end{pmatrix} 
    \end{align}
  \end{mdframed}
Alternativ: \\
Gleichungen einzeln Linearisieren
  \begin{mdframed}[style=exercise]
    \begin{align}
        \delta \dot{x_1} = \frac{\partial f_1}{\partial x_1} \big|_{x_{1R}} \cdot \delta x_1
        &+ \frac{\partial f_1}{\partial x_2} \big|_{x_{2R}} \delta x_2+ \delta u_1 \\
        &...
    \end{align}
  \end{mdframed}

\subsubsection{Stabilität}
System stabil wenn
\begin{enumerate}
    \item alle EW $\lambda_k$ neg. Realteil $\rightarrow$ \textbf{asymptotisch stabil}
    \item $Re(\lambda_k)=0$ $\rightarrow$ stabil falls \textbf{alg VFH = geo VFH}
\end{enumerate}
Analyse der Jacobi-Matrix mit Routh-Hurwitz-Kriterium


\end{document}
