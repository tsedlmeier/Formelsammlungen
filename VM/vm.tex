\documentclass[10pt,a4paper]{article}
\usepackage[utf8]{inputenc}
\usepackage[ngerman]{babel}
\usepackage[T1]{fontenc}
\usepackage{amsmath}
\usepackage{amsfonts}
\usepackage{amssymb}
\usepackage{graphicx}
\usepackage{lmodern}
\usepackage{physics}
\usepackage[left=1cm,right=1cm,top=1.5cm,bottom=1.2cm]{geometry}
\usepackage{siunitx}
\usepackage{fancyhdr}
\usepackage{enumerate}
\usepackage{mhchem}
\usepackage{mathtools}
\usepackage{graphicx}
\usepackage{float}
\usepackage[table]{xcolor}
\usepackage{mdframed}
\usepackage{csquotes}
\usepackage{trfsigns}
\usepackage{capt-of}
\usepackage{adjustbox}
\usepackage{verbatim}

\sisetup{locale=DE}
\sisetup{per-mode = symbol-or-fraction}
\sisetup{separate-uncertainty=true}
\DeclareSIUnit\year{a}
\DeclareSIUnit\clight{c}
\mdfdefinestyle{exercise}{
	backgroundcolor=black!10,roundcorner=8pt,hidealllines=true,nobreak
}

\begin{document}
\twocolumn
\pagestyle{fancy}
% \lhead{DSV Formelsammlung, Stand {\input{\string"| date + " %Y-%d-%m" \string"}}}
\lhead{Formelsammlung Mathe, \today}
\rhead{Sedlmeier, Toni}
\section{Begriffe - Linearität und Zeitinvarianz}
%%%%%%%%%%%%%%%%%%%%%%%%%%%%%%%%%%%%% Elementare System-Eigenschaften %%%%%%%%%%%%%%%%%%%%%%%%%%%%%%%%%%%%%%%%%%%%
  \subsection{Linear/Nichtlinear}
  Definition \textbf{Lineares} System:
  \begin{mdframed}[style=exercise]
    \begin{align}
        u = \alpha_1 u_1(t)+ \alpha_2 u_2(t)\rightarrow y = \alpha_1 y_1(t)+ \alpha_2 y_2(t) \\  
    \end{align}
  \end{mdframed}
  Nichtlinearitäten: 
  \begin{itemize}
    \item u(t) ist in nichtlin. Funktion $f$ (z.B \sin(), $\sqrt$, $\e$) versteckt
    \item Addition mit Konstante $k$
  \end{itemize}
  \begin{mdframed}[style=exercise]
    \begin{align}
        u(t)\rightarrow y(t) &= f\{u(t)\} \\
        u(t)\rightarrow y(t) &= u(t) + k
    \end{align}
  \end{mdframed}

  \subsection{Zeitvariant/Zeitinvariant}
  Definition \textbf{Zeitinvariantes} System:
  \begin{mdframed}[style=exercise]
    \begin{align}
        u(t-\tau) \rightarrow y(t-\tau)
    \end{align}
  \end{mdframed}
  Zeitvarianz: 
  \begin{itemize}
    \item Multiplikation mit $f(t)$
    \item Zeitverschiebung von \textbf{nur} $u(t)$ bzw. $y(t)$
  \end{itemize}
  \begin{mdframed}[style=exercise]
    \begin{align}
        u(t) &\rightarrow y(t) = f(t)u(t)\\
        u(t) &\rightarrow y(t) = u(t-t_0)\\
        u(t) &\rightarrow y(t-t_0) = u(t)
    \end{align}
  \end{mdframed}

\subsection{DGL^n $\rightarrow$ DGL-System^1}
Eine DGL n-ter Ordnung kann in ein DGL-System n-ter Ordnung umgewandelt werden:
  \begin{mdframed}[style=exercise]
    \begin{align}
        a_n y^{(n)}(t) + ... + a_2 y^{''} + a_1 y^{'} = b_0u(t)
    \end{align}
  \end{mdframed}
Lösungsansatz für n=4: 
  \begin{mdframed}[style=exercise]
    \begin{align}
        x_1(t) &= y(t) & \indent      x_1^{'}(t) &= x_2(t)\\
        x_2(t) &= y^{'}(t) &\indent   x_2^{'}(t) &= x_3(t)\\
        x_3(t) &= y^{''}(t) &\indent  x_3^{'}(t) &= x_4(t)\\
        x_4(t) &= y^{(3)}(t) &\indent 
    \end{align}
  \end{mdframed}
  \begin{mdframed}[style=exercise]
    \begin{align}
        x_4^{'}(t) &= \frac{1}{a_4}[u(t)-a_3x_3(t)-a_2x_2(t)-a_1x_1(t)]
    \end{align}
  \end{mdframed}
Zustandsraumdarstellung:
  \begin{mdframed}[style=exercise]
    \begin{align}
        \begin{pmatrix}
        x_1^{'}(t) \\
        x_2^{'}(t) \\
        x_3^{'}(t) \\
        x_4^{'}(t) \\
        \end{pmatrix} 
        =
        \begin{pmatrix}
            0 & 1 & 0 & 0 \\
            0 & 0 & 1 & 0 \\
            0 & 0 & 0 & 1 \\
            -\frac{a_1}{a_4} & -\frac{a_2}{a_4} & -\frac{a_2}{a_4} &  -1 \\
        \end{pmatrix}
        \begin{pmatrix}
        x_1(t) \\
        x_2(t) \\
        x_3(t) \\
        x_4(t) \\
        \end{pmatrix} 
        +
        \begin{pmatrix}
        0 \\
        0 \\
        0 \\
        \frac{b_0}{a_4}u(t) \\
        \end{pmatrix} 
    \end{align}
  \end{mdframed}


\section{Matrizenrechnung}
\subsection{Eigenwerte und Eigenvektoren}
Eine Zahl $\lambda \in \mathbb{C}$ heißt Eigenwert von Matrix $A$($n$x$n$),
falls die Multiplikation von $A$ mit Vektor $v$ $\in \mathbb{C}^n$ in eine Skalarmultiplikation
mit $\lambda$ zusammenfällt. Dieser Vektor $v$ wird dann Eigenvektor genannt.
  \begin{mdframed}[style=exercise]
    \begin{align}
        Av = \lambda v
    \end{align}
  \end{mdframed}
Vorteil: Es müssen nur noch $n$-Teilmultiplikationen durchgeführt werden.
$O(n^2) \rightarrow O(n)$

\subsection{Eigenwertprobleme}
EW = Nullstellen des char. Polynoms $p_A(\lambda)$.\\
  \begin{mdframed}[style=exercise]
    \begin{align}
        det(A-\lambda E) \Longleftrightarrow 
        \begin{vmatrix}
             a_{11}-\lambda & a_{12} & ...  & a_{1n} \\
             a_{21}& a_{22}-\lambda & ... & a_{2n} \\
              ... & ...  & ...  & a_{3n} \\
             a_{n1} & a_{n2} & ... &  a_{nn}-\lambda\\
        \end{vmatrix} = 0
    \end{align}
  \end{mdframed}
  \begin{mdframed}[style=exercise]
    \begin{align}
        p_A(\lambda) &= a_n \lambda^n +...+ a_1 \lambda + a_0\\
        p_A(\lambda) &= (\lambda - \lambda_1)(\lambda - \lambda_2)...(\lambda - \lambda_n)
    \end{align}
  \end{mdframed}

  \begin{itemize}
    \item $p_A(\lambda)$ hat $n$ Nulltellen
    \item VFH der NS $\lambda_i$ = \textbf{algebraische} VFH des EW $a(\lambda_i)$ 
    \item $\sum_i a(\lambda_i) = n$ 
    \item falls A \textbf{symmetrisch} $\rightarrow$ alle EW sind \textbf{reel}
  \end{itemize}
Für Eigenvektoren einfach die EW auf der Hauptdiagonalen abziehen
  \begin{mdframed}[style=exercise]
    \begin{align}
        (A-\lambda E)v = 0
    \end{align}
  \end{mdframed}
Bestimmung des i-ten EV $v_i$ mit zugehörigem EW $\lambda_i$ \\
Bsp für n = 2: (v bestimmen durch Hinschauen)
  \begin{mdframed}[style=exercise]
    \begin{align}
        \lambda_i : 
        \begin{pmatrix}
            a_{11}-\lambda_i & a_{12} \\
            a_{21} & a_{22}-\lambda_i \\
        \end{pmatrix} v = 0
    \end{align}
  \end{mdframed}

  \begin{itemize}
    \item \textbf{geometrische} VFH $g(\lambda_i)$ = Anzahl lin. unabh. EV $v$
  \end{itemize}

  \subsection{Geometrische Lage von EW}
  EW-Bestimmung teilweise aufwendig. Aussage über Stabilität von DGL-Systemen anhand der Lage der EW der Systemmatrix A 
  (z.B alle EW links in komplexer Halbebene).\\
  Kriterien:
  \begin{itemize}
    \item Gerschgorin-Kreise 
    \item Routh-Hurwitz-Kriterium
  \end{itemize}

  \subsubsection{Gerschgorin-Kreise}
Aussage über geometrische Lage der EW einer Matrix.
Ablesen von Systemeigenschaften ohne tatsächliche Berechnung der EW.\\
Vorgehen:
  \begin{itemize}
    \item Für jede Zeile ein Kreis
    \item Diagonalelement ist der Mittelpunkt
    \item Summe der Beträge der Restelemente der Radius
  \end{itemize}
Bsp für n = 3
  \begin{mdframed}[style=exercise]
    \begin{align}
        \begin{pmatrix}
             a_{11} & a_{12} & a_{13} \\
             a_{21} & a_{22} & a_{23} \\
             a_{31} & a_{32} & a_{33} 
        \end{pmatrix}
    \end{align}
  \end{mdframed}
  dann
  \begin{mdframed}[style=exercise]
    \begin{align}
        K1: MP=(a_{11}/0), R=|a_{12}+a_{13}|\\
        K1: MP=(a_{22}/0), R=|a_{21}+a_{23}|\\
        K1: MP=(a_{33}/0), R=|a_{31}+a_{32}|
    \end{align}
  \end{mdframed}
  \begin{itemize}
    \item Disjunkte Kreise enthalten genau einen EW
    \item Vereinigung von $m$ disjunkten Kreisen enthält $m$ EW
  \end{itemize}

  \subsection{s-Übertragungsfunktion}
  \begin{mdframed}[style=exercise]
    \begin{align}
        G(s)=\frac{Y(s)}{U(s)}= \frac{b_n s^{n}+...+b_1s+b_0 } {s^{n}+...+a_1s+a_0 }
    \end{align}
  \end{mdframed}
  Laplace-Transformation ist nur für lineare, zeitinvariante Systeme definiert. (Superpositionsprinzip!)

  \section{Integraltransformation}
  Fourier-und Laplace-Transformation sind Integraltransformationen. 
  Eine Transformation ist ein Operator $T$, der eine Funktion $f$ aus einem Funktionsraum $F$ 
  auf eine Funktion aus anderen Funktionsraum $G$ abbildet
  \begin{mdframed}[style=exercise]
    \begin{align}
        T:\left\{\begin{array}{ll} F \rightarrow G, \\
         f	\mapsto Tf. & \end{array}\right. 
    \end{align}
  \end{mdframed}
  Eine Integraltransformation ist lediglich eine Transformation, in die ein Integral verwickelt ist.
  Wir definieren eine neue Laufvariable $\eta$. Der $K(\eta,t)$ ist der Kern der Integraltransformation. 
  Dieser beschreibt die Beziehung zwischen der ursprünglichen Funktion $f(t)$ und der transformierten Funktion $Tf(\eta)$.
  Bspw. ist der Kern der Fouriertransformation $K(f,t) = e^{-j2\pi ft}$. \\
  \textbf{Allgemeine Definition}
  \begin{mdframed}[style=exercise]
    \begin{align}
        Tf(\eta) = \displaystyle\int K(\eta,x) f(x) dx
    \end{align}
  \end{mdframed}
  
  \subsection{Eigenfunktion}
  Eine Funktion $x(t)$ ist zu einem Operator $T$ eine Eigenfunktion, falls die Anwedung von T auf x folgende Form hat:
  \begin{mdframed}[style=exercise]
    \begin{align}
        Tx(t) = \lambda x(t) \ \ mit \ \ \lambda \in \mathbb{C}
    \end{align}
  \end{mdframed}
  Bsp: Exponentialfunktion ist Eigenfunktion bzgl. Ableitungsoperator D:
    \begin{align*}
        D\{e^{\textcolor{red}\lambda t}\} = \frac{d}{dt}\{e^{\textcolor{red}\lambda t}\} = \textcolor{red} \lambda e^{\lambda t}
    \end{align*}
  Da sinusförmige Eingangssignale der Form 
    \begin{align*}
        s_e(t) = e^{j2\pi ft} = cos(2\pi ft) + j sin (2\pi ft)
    \end{align*}
  LTI-Systeme ohne Formänderung passieren sind diese Eigenfunktionen von LTI-Systemen.\\
  \textbf{Beispiel:}\\
  Ein sinusförmiges Eingangssignal $x(t)$ wird auf ein System $S$ gegeben und ruft das Ausgangssignal $y(t)$ hervor.
  \begin{mdframed}[style=exercise]
    \begin{align*}
        x(t) &= e^{j\omega t} = cos(\omega t) + j sin (\omega t) \ \ mit \ \ \omega = 2\pi f\\
        y(t) &= S\{x(t)\} = S\{e^{j\omega t}\} 
    \end{align*}
  \end{mdframed}
  Da es sich um ein LTI-System handelt ist dessen Operator $S$ linear und zeitinvariant.\\ 
  Es gilt Zeitinvarianz:
  \begin{mdframed}[style=exercise]
    \begin{align*}
        x(t-\tau) &= e^{j\omega (t-\tau)}\\
        y(t-\tau) &= S\{x(t-\tau)\} = S\{e^{j\omega (t-\tau)}\} = S\{e^{-j\omega\tau} e^{j\omega t}\}
    \end{align*}
  \end{mdframed}
  Weiterhin folgt aus Linearit das Superpositionsprinzip:
  \begin{mdframed}[style=exercise]
    \begin{align*}
        y(t-\tau) &= S\{e^{-j\omega\tau} e^{j\omega t}\} = S\{ \underbrace{e^{-j\omega\tau}}_{ \ne f(t) }  \} \cdot S\{e^{j\omega t}\} \\
        y(t-\tau) &= e^{-j\omega\tau} \cdot S\{e^{j\omega t}\} = \underbrace{e^{-j\omega\tau}}_{ = \lambda } \cdot S\{x(t) \} = \lambda \cdot y(t)
    \end{align*}
  \end{mdframed}
  Es gilt also 
  \begin{mdframed}[style=exercise]
    \begin{align*}
        y(t) = S\{x(t)\} = \lambda y(t) 
    \end{align*}
  \end{mdframed}
  

  \subsection{Verallgemeinerung der Eigenfunktionen}
  Wir haben gesehen, dass komplexe Exponentialfunktionen der Form $e^{-j2\pi ft}$ Eigenfunktionenen von LTI-Systemen sind.
  Leider sind reale Signale im Allgemeinen keine Eigenfunktionen von LTI-Systemen.
  Wir können aber beliebige Signale $x(t)$ als eine Überlagerung von Eigenfunktionen $e^{-j2\pi ft}$ mit unterschiedlichen
  Frequenzen $\omega = 2\pi f$ darstellen.
  Wir bilden eine Integraltransformation, bei der der Kern $K(j2\pi f,t)$ eine Funktion der Form $e^{-j2\pi ft}$ hat.\\
  mit $\omega = 2\pi f$ ergiebt sich:
  \begin{mdframed}[style=exercise]
    \begin{align*}
        F(j\omega) &= \mathbb{F}f(j\omega) = \displaystyle\int K(j\omega,t) f(t) dt\\
        F(j\omega) &= \displaystyle\int  e^{-j\omega t} f(t) dt
    \end{align*}
  \end{mdframed}
  Wir erhalten die also die Fouriertransformation.
  Der Term $e^{-j \omega t}$ bewirkt eine Drehung, die nicht gegen 0 geht. Die Fouriertransformation ist über das uneigentliche Integral definiert.
  Damit das Integral konvergiert und die Fouriertransformation existiert muss $\lim_{t\rightarrow\infty} f(t)=0$ gelten.
  Daher wollen wir die Fouriertransformation verallgemeinern, indem der Kern der Fouriertransformation um 
  eine abklingede Exponentialfunktion $e^{-\sigma} $ mit $ \sigma \in \mathbf{R}$ erweitert wird.
  \begin{mdframed}[style=exercise]
    \begin{align*}
        s = \sigma + j \omega \rightarrow e^{-st} = e^{-t(\sigma + j \omega) } = e^{-\sigma t} \cdot e^{-j \omega t}
    \end{align*}
  \end{mdframed}
  Wir erhalten die Laplace-Transformation
  \begin{mdframed}[style=exercise]
    \begin{align*}
        F(j\omega) &= \displaystyle\int  e^{-\sigma t} \cdot e^{-j\omega t} f(t) dt = \displaystyle\int  e^{-st} f(t) dt
    \end{align*}
  \end{mdframed}

\end{document}
